\documentclass[Lau, binding=0.6cm, twoside]{sapthesis}
\usepackage [italian]{babel}
\usepackage[T1]{fontenc}
\usepackage[utf8]{inputenc}
\usepackage{microtype}
\usepackage{amsmath}
\usepackage{amssymb}
\usepackage{graphicx}
\usepackage{hyperref}
\hypersetup{pdftitle={metti il titolo della tesi},pdfauthor={Davide Albano}}

\title{rimetti il titolo}
\author{Davide Albano}
\IDnumber{metti la matricola}
\course{Ingegneria Informatica e Automatica}
\courseorganizer{Facoltà di ..}
\AcademicYear{2020/2021}
\copyyear{2021}
\advisor{Prof. Giorgio Grisetti}
\authoremail{metti la tua email}

\examdate{ }
\examiner{ }
\versiondate{ } %queste tre righe non serve che le compili poi quando vai a stampare fai togliere questa parte perchè non si può cancellare sennò non parte il file

\begin{document}
\maketitle
\dedication{scrivi qui la dedica se la vuoi senno togli la riga del comando}
\begin{abstract}
scrivi qui il sommario se serve
\end{abstract}
\tableofcontents

\chapter{metti il titolo del capitolo}
\section{se serve metti il titolo del paragrafo}
qui poi inizia a scrivere il paragrafo, ogni volta che devi aggiungere un capitolo o un paragrafo copi il comando di sopra. 
Per andare a capo devi mettere \\ 
Il comando che trovi sotto indica la fine del documento tutto ciò che devi scrivere va sopra.

\end{document}