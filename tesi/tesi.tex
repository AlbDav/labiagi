\documentclass[Lau, binding=0.6cm, twoside]{sapthesis}
\usepackage [italian]{babel}
\usepackage[T1]{fontenc}
\usepackage[utf8]{inputenc}
\usepackage{microtype}
\usepackage{amsmath}
\usepackage{amssymb}
\usepackage{graphicx}
\usepackage{hyperref}
\hypersetup{pdftitle={Sviluppo di un sistema di Collision Avoidance tramite il metodo dell'Artificial Potential Field},pdfauthor={Davide Albano}}
\setlength{\parindent}{1.25em}
\setlength{\parskip}{0.75em}

\title{Sviluppo di un sistema di Collision Avoidance tramite il metodo dell'Artificial Potential Field}
\author{Davide Albano}
\IDnumber{1708530}
\course{Ingegneria Informatica e Automatica}
\courseorganizer{Facoltà di Ingegneria dell'Informazione, Informatica e Statistica}
\AcademicYear{2020/2021}
\copyyear{2021}
\advisor{Prof. Giorgio Grisetti}
\authoremail{albano.1708530@studenti.uniroma1.it}

\examdate{ }
\examiner{ }
\versiondate{ } %queste tre righe non serve che le compili poi quando vai a stampare fai togliere questa parte perchè non si può cancellare sennò non parte il file
\begin{document}
\maketitle
\begin{abstract}
L'obbiettivo di questo lavoro è l'implementazionie di un sistema che permetta a un robot di muoversi all'interno di un ambiente evitando eventuali ostacoli.
Il robot preso in considerazione non conosce l'ambiente in cui dovrà muoversi e può quindi basarsi soltanto sulle rilevazioni fatte tramite in laser scanner.

Per l'implementazione è stato usato ROS (Robot Operating System) un framework open source usato per gestire le operazioni e le cominucazioni dei vari nodi di un robot.
Per la simulazione è stato usato il pacchetto stage\_ros, mentre per la visualizzazione delle informazioni sulla traiettoria calcolate dal robot è stato usato il tool RVIZ.

Data l'assenza di informazioni iniziali riguardo alla mappa è stata necessario scegliere un algoritmo di local motion planning.
In particolare è stato scelto il metodo dell'Artificial Potential Field, che è spesso usato per la sua semplicità sia dal punto di vista dello sviluppo sia dal punto di vista computazionale.
\end{abstract}
\tableofcontents

\chapter{Introduzione}
\section{Formulazione problema}
Quando si sviluppa un sistema di navigazione autonoma per un robot mobile può capitare che le informazioni sull'ambiente in cui il robot dovrà muoversi siano scarse, non aggiornate o mancanti.
Questo porta alla necessita di sviluppare algoritmi di local motion, che permettono al robot di calcolare una traiettoria in grado di evitare gli ostacoli incontrati durante la navigazione.

\end{document}